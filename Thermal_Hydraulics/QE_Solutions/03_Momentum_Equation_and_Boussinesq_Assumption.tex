%% LyX 2.1.3dev created this file.  For more info, see http://www.lyx.org/.
%% Do not edit unless you really know what you are doing.
\documentclass[english]{article}
\usepackage[T1]{fontenc}
\usepackage[latin9]{inputenc}
\usepackage[letterpaper]{geometry}
\geometry{verbose,tmargin=2cm,bmargin=2cm,lmargin=2cm,rmargin=2cm,headheight=1cm,headsep=1cm,footskip=1cm}
\usepackage{fancyhdr}
\pagestyle{fancy}
\setcounter{secnumdepth}{-1}
\usepackage{amsmath}
\usepackage{amssymb}
\usepackage{cancel}

\makeatletter
%%%%%%%%%%%%%%%%%%%%%%%%%%%%%% User specified LaTeX commands.
\usepackage{pgf}
\usepackage{lastpage}

\let\oldmaketitle\maketitle

\renewcommand{\maketitle}{\oldmaketitle \thispagestyle{fancy}}

\makeatother

\usepackage{babel}
\begin{document}

\lhead{Alex Hagen}


\chead{QE Studying: Thermal Hydraulics Solutions}


\rhead{1/4/15}


\cfoot{\thepage\ of \pageref{LastPage}}


\title{Momentum Equation and Boussinesq Assumption \cite[pp. 75-78]{Bird2007}}

\maketitle

\section{Momentum Equation}

Applying the the GBE $\psi=\rho\vec{v}$, $\boldsymbol{J}=\boldsymbol{T}=p\boldsymbol{I}+\boldsymbol{\tau}$,
and $\dot{\psi_{g}}=\rho\vec{g}$, we get

\[
\underbrace{\frac{\partial\left(\rho\vec{v}\right)}{\partial t}}_{\substack{\text{rate of change}\\
\text{of momentum}\\
\text{per unit volume}
}
}+\underbrace{\nabla\cdot\left(\rho\vec{v}\vec{v}\right)}_{\substack{\text{momentum change}\\
\text{by convection}\\
\text{per unit volume}
}
}=-\underbrace{\nabla p}_{\substack{\text{pressure}\\
\text{force}
}
}-\underbrace{\nabla\cdot\boldsymbol{\tau}}_{\substack{\text{viscous}\\
\text{force}
}
}+\underbrace{\rho\vec{v}}_{\substack{\text{gravity}\\
\text{force}
}
}
\]


By using the substantial derivative, this become's Cauchy's Equation
of Motion: $\rho\frac{D\vec{v}}{Dt}=-\nabla p-\nabla\cdot\boldsymbol{\tau}+\rho\vec{g}$,
which is idential in form to Newton's Second Law of Motion: $\rho\vec{a}=\sum\vec{F}$.


\section{Boussinesq Assumption}

The boussinesq assumption is used in flows where gravity is important:
\begin{itemize}
\item Thermal expansion causes density change.
\item This happens by the thermal expansion coefficient, $\beta$, 
\[
\beta\equiv\frac{1}{\nu}\left.\frac{\partial\nu}{\partial T}\right)_{p}=\frac{1}{\rho}\left.\frac{\partial\rho}{\partial T}\right)_{p}
\]

\item This density change is only important in gravity term.
\end{itemize}
Applying, we get

\[
dp=-\rho\beta dT\;\;\therefore\;\;\rho-\overline{\rho}=-\overline{\rho}\beta\left(T-\overline{T}\right)
\]


and putting into the momentum equations

\[
\overline{\rho}\frac{D\vec{v}}{Dt}=-\cancelto{\nabla p_{dyn}}{\nabla p}-\nabla\cdot\boldsymbol{\tau}+\left[\cancelto{0}{\overline{\rho}}-\overline{\rho}\beta\left(T-\overline{T}\right)\right]\vec{g}
\]


\bibliographystyle{plain}
\bibliography{th}

\end{document}
