%% LyX 2.1.3dev created this file.  For more info, see http://www.lyx.org/.
%% Do not edit unless you really know what you are doing.
\documentclass[english]{article}
\usepackage[T1]{fontenc}
\usepackage[latin9]{inputenc}
\usepackage[letterpaper]{geometry}
\geometry{verbose,tmargin=2cm,bmargin=2cm,lmargin=2cm,rmargin=2cm,headheight=1cm,headsep=1cm,footskip=1cm}
\usepackage{fancyhdr}
\pagestyle{fancy}
\setcounter{secnumdepth}{-1}
\usepackage{float}
\usepackage{amsmath}
\usepackage{cancel}
\usepackage{esint}

\makeatletter
%%%%%%%%%%%%%%%%%%%%%%%%%%%%%% User specified LaTeX commands.
\usepackage{pgf}
\usepackage{tikz}
\usepackage{pstricks}
\usepackage{lastpage}

\makeatletter

\def\maketitle{%
\lhead{\@author}
\chead{\textsc{\@title}}
\rhead{\@date}
\rfoot{\thepage\ of \pageref{LastPage}}
\lfoot{QE Studying: Thermal Hydraulics Solutions}
\cfoot{}
\thispagestyle{fancy}
}

\makeatother

\makeatother

\usepackage{babel}
\begin{document}

\title{Control Volume Analysis}


\author{Alex Hagen}


\date{1/19/15}

\maketitle

\section{Balance Equations}

\[
\frac{d}{dt}\int_{CV}\rho dV-\oint_{CS}\rho\vec{v}\cdot\vec{n}dA=\left(\rho\vec{v}A\right)_{in}-\left(\rho\vec{v}A\right)_{out}=\sum_{i}\dot{m_{i}}
\]


\[
\frac{d}{dt}\int_{CV}\rho\vec{v}dV=\sum\vec{F}+\sum_{i}\dot{m_{i}}\vec{v}
\]


\[
\frac{d}{dt}\int_{CV}\rho\vec{e}dV=\dot{Q}+W_{s}+\sum_{i}\left(\vec{e_{i}}+\frac{p_{i}}{\rho_{i}}\right)
\]


\[
\vec{e}\equiv u+\cancelto{0}{\left(\frac{\vec{v}^{2}}{2}+\vec{g}z\right)}
\]


and for a plant

\[
\dot{Q}_{core}+W_{sp1}-\left(\dot{Q}_{SG}+\dot{Q}_{loss}\right)+\sum_{i}\dot{m}_{i}\left(\vec{e_{i}}+\frac{p_{i}}{\rho_{i}}\right)<0
\]


For reference, when the reactor scrams, the power profile can be assumed
to drop to $8\%$ immediately, and stay there for a long time, or:

\[
P_{d}\left(t\right)=0.062P_{0}\left[t^{-0.2}-\left(t_{0}+t\right)^{-0.2}\right]\;\;\text{Way and Winger}
\]


When the pump trips, the pressure follows a coast down, which can
be solved using the momentum equation, which follows:

\begin{figure}[H]
\begin{centering}

\definecolor{ccccccc}{RGB}{204,204,204}


\begin{tikzpicture}[y=0.80pt, x=0.8pt,yscale=-1, inner sep=0pt, outer sep=0pt]
\begin{scope}[shift={(0,-782.35975)}]
  \path[color=black,fill=ccccccc,line width=1.600pt] (155.0313,811.9063) --
    (157.0313,811.9063) -- (157.0313,795.0938) -- (155.0313,795.0938) --
    cycle(155.0313,859.9063) -- (157.0313,859.9063) -- (157.0313,835.9063) --
    (155.0313,835.9063) -- cycle(155.0313,907.9063) -- (157.0313,907.9063) --
    (157.0313,883.9063) -- (155.0313,883.9063) -- cycle(155.0313,955.9063) --
    (157.0313,955.9063) -- (157.0313,931.9063) -- (155.0313,931.9063) --
    cycle(155.0313,1003.9062) -- (157.0313,1003.9062) -- (157.0313,979.9063) --
    (155.0313,979.9063) -- cycle;
  \path[fill=ccccccc] (199.85852,1044.8153) node[above right] (text3932) {Time
    ($t$) [s]};
  \path[cm={{0.0,-1.0,1.0,0.0,(0.0,0.0)}},fill=ccccccc] (-977.02356,19.215298)
    node[above right] (text3932-1) {\rotatebox{90}{Pressure Difference ($\Delta
    p$) [Pa]}};
  \path[fill=ccccccc] (36.135246,903.2829) node[above right] (text3847-2-5-2-1)
    {$0$};
  \path[draw=ccccccc,line join=miter,line cap=butt,miter limit=4.00,line
    width=1.600pt] (46.5621,899.3597) -- (354.3550,899.3597)(46.4594,794.8260) --
    (354.5406,794.8260) -- (354.5406,1003.8935) -- (46.4594,1003.8935) -- cycle;
  \path[shift={(0,782.35975)},draw=black,line join=miter,line cap=butt,line
    width=0.800pt] (46.2959,35.6805) -- (156.3550,35.6805) .. controls
    (156.3550,107.2331) and (199.5087,116.8935) .. (271.7101,116.8935) --
    (271.7101,168.3728) -- (353.8107,169.2604);
  \path[fill=ccccccc] (149.74971,1020.1519) node[above right] (text3847-2-5-2-1-4)
    {$t_{\text{trip}}$};
  \begin{scope}[fill=ccccccc]
    \path[shift={(0,782.35975)},color=black,fill=ccccccc,line width=1.200pt]
      (300.0000,115.3438) -- (296.3750,125.1875) .. controls (297.2458,124.5625) and
      (298.2388,124.1885) .. (299.2500,124.0625) -- (299.2500,133.8125) --
      (300.7500,133.8125) -- (300.7500,124.0625) .. controls (301.7579,124.1869) and
      (302.7394,124.5589) .. (303.5938,125.1875) -- (300.0000,115.3438) --
      cycle(299.2500,152.7813) -- (299.2500,161.1875) .. controls
      (298.2428,161.0629) and (297.2600,160.7220) .. (296.4063,160.0938) --
      (300.0000,169.9063) -- (303.6250,160.0938) .. controls (302.7548,160.7183) and
      (301.7605,161.0613) .. (300.7500,161.1875) -- (300.7500,152.7813) --
      (299.2500,152.7813) -- cycle;
  \end{scope}
  \path[fill=ccccccc] (279.53284,929.38678) node[above right]
    (text3847-2-5-2-1-4-8) {$\Delta p_{\text{flywheel}}$};
\end{scope}

\end{tikzpicture}

\par\end{centering}

\protect\caption{Pump Coast Down Pressure}


\end{figure}



\subsection{LOCA}

In a LOCA, the pump trips immediately, and (hopefully) the reactor
scrams. Then, at a later time, the ECCS turn on. The energy being
lost in the summation term in the energy equation is high whereas
the energy added in from the ECCS is low quality, so it works.

\[
\frac{d}{dt}\int_{CV}\rho dV=-\left(\rho vA\right)_{break}+\underbrace{\left(\rho vA\right)_{eccs}}_{\substack{\text{after some}\\
\text{time }t
}
}\underbrace{>0}_{\substack{\text{ECCS}\\
\text{design}\\
\text{requirement}
}
}
\]


\[
\frac{d}{dt}\int_{CV}\rho edV=\cancelto{\dot{Q}_{decay}}{\dot{Q}_{core}}+\cancelto{W_{cd}}{W_{sp1}}-\left(\dot{Q}_{SG}+\dot{Q}_{loss}\right)<0
\]



\subsection{LOFA}

In a LOFA, there is a pump trip and a scram, and since the steam generator
has nothing really exchanging through it, it only can remove the heat
through natural convection.

\[
\frac{d}{dt}\int_{CV}\rho edV=\cancelto{\dot{Q}_{decay}}{\dot{Q}_{core}}+\cancelto{W_{cd}}{W_{sp1}}-\left(\cancelto{\dot{Q}_{NC}}{\dot{Q}_{SG}}+\dot{Q}_{loss}\right)<0
\]



\subsection{LOHSA}

In a LOHSA, the secondary loop fails, so there is only a small amount
of heat able to be removed by the steam generator.

\[
\frac{d}{dt}\int_{CV}\rho edV=\cancelto{\dot{Q}_{decay}}{\dot{Q}_{core}}+W_{sp1}-\left(\cancelto{\text{small}}{\dot{Q}_{SG}}+\dot{Q}_{loss}\right)<0
\]


For this accident, the HPIS is required.


\subsection{SBO}

For a station blackout, almost all components fail, including the
secondary loop, so all drop to their lower values.

\[
\frac{d}{dt}\int_{CV}\rho edV=\cancelto{\dot{Q}_{decay}}{\dot{Q}_{core}}+\cancelto{W_{cd}}{W_{sp1}}-\left(\cancelto{\text{small}}{\dot{Q}_{SG}}+\dot{Q}_{loss}\right)<0
\]


For this accident, the HPIS are required, but there is no power to
turn it on.

\bibliographystyle{plain}
\bibliography{th}

\end{document}
