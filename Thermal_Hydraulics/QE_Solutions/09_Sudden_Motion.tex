%% LyX 2.1.3dev created this file.  For more info, see http://www.lyx.org/.
%% Do not edit unless you really know what you are doing.
\documentclass[english]{article}
\usepackage[T1]{fontenc}
\usepackage[latin9]{inputenc}
\usepackage[letterpaper]{geometry}
\geometry{verbose,tmargin=2cm,bmargin=2cm,lmargin=2cm,rmargin=2cm,headheight=1cm,headsep=1cm,footskip=1cm}
\usepackage{fancyhdr}
\pagestyle{fancy}
\setcounter{secnumdepth}{-1}
\usepackage{float}
\usepackage{amsmath}
\usepackage{amssymb}
\usepackage{cancel}

\makeatletter
%%%%%%%%%%%%%%%%%%%%%%%%%%%%%% User specified LaTeX commands.
\usepackage{pgf}
\usepackage{tikz}
\usepackage{pstricks}
\usepackage{lastpage}

\let\oldmaketitle\maketitle

\renewcommand{\maketitle}{\oldmaketitle \thispagestyle{fancy}}

\makeatother

\usepackage{babel}
\begin{document}

\lhead{Alex Hagen}


\chead{QE Studying: Thermal Hydraulics Solutions}


\rhead{1/10/14}


\cfoot{\thepage\ of \pageref{LastPage}}


\title{Sudden Motion \{Bird\}}

\maketitle

\subsection{Illustration}

\begin{figure}[H]
\begin{centering}

\definecolor{ccccccc}{RGB}{204,204,204}


\begin{tikzpicture}[y=0.80pt, x=0.8pt,yscale=-1, inner sep=0pt, outer sep=0pt]
\begin{scope}[shift={(0,-782.35975)}]
  \begin{scope}[shift={(-2.0,0)}]
    \path[shift={(0,782.35975)},rounded corners=0.0000cm] (11.7108,54.6506)
      rectangle (260.2410,67.9880);
    \path[shift={(0,782.35975)},draw=black,line width=0.800pt] (17.5661,54.6506) --
      (11.7108,60.5058)(12.7140,67.9880) -- (26.0513,54.6506)(34.5366,54.6506) --
      (21.1993,67.9880)(29.6846,67.9880) -- (43.0219,54.6506)(51.5072,54.6506) --
      (38.1698,67.9880)(46.6551,67.9880) -- (59.9925,54.6506)(68.4777,54.6506) --
      (55.1404,67.9880)(63.6257,67.9880) -- (76.9630,54.6506)(85.4483,54.6506) --
      (72.1110,67.9880)(80.5962,67.9880) -- (93.9336,54.6506)(102.4189,54.6506) --
      (89.0815,67.9880)(97.5668,67.9880) -- (110.9042,54.6506)(119.3894,54.6506) --
      (106.0521,67.9880)(114.5374,67.9880) -- (127.8747,54.6506)(136.3600,54.6506)
      -- (123.0227,67.9880)(131.5079,67.9880) --
      (144.8453,54.6506)(153.3306,54.6506) -- (139.9932,67.9880)(148.4785,67.9880)
      -- (161.8158,54.6506)(170.3011,54.6506) --
      (156.9638,67.9880)(165.4491,67.9880) -- (178.7864,54.6506)(187.2717,54.6506)
      -- (173.9343,67.9880)(182.4196,67.9880) --
      (195.7570,54.6506)(204.2422,54.6506) -- (190.9049,67.9880)(199.3902,67.9880)
      -- (212.7275,54.6506)(221.2128,54.6506) --
      (207.8755,67.9880)(216.3607,67.9880) -- (229.6981,54.6506)(238.1834,54.6506)
      -- (224.8460,67.9880)(233.3313,67.9880) --
      (246.6687,54.6506)(255.1539,54.6506) -- (241.8166,67.9880)(250.3019,67.9880)
      -- (260.2410,58.0489)(260.2410,66.5341) -- (258.7871,67.9880);
  \end{scope}
  \path[shift={(0,782.35975)},rounded corners=0.0000cm] (11.3855,54.6506)
    rectangle (258.2892,68.3133);
  \path[draw=black,line join=miter,line cap=butt,line width=0.800pt]
    (11.5482,850.3477) -- (258.4518,850.3477);
  \begin{scope}[shift={(0,782.35975)},fill=black]
  \end{scope}
  \path[shift={(0,782.35975)},draw=ccccccc,line join=miter,line cap=butt,line
    width=0.800pt] (85.0663,71.8916) -- (194.3675,71.8916);
  \path[shift={(0,782.35975)},draw=ccccccc,line join=miter,line cap=butt,line
    width=0.800pt] (205.2651,71.8916) -- (257.0097,71.8916);
  \path[shift={(0,782.35975)},draw=ccccccc,line join=miter,line cap=butt,line
    width=0.800pt] (8.4698,74.8193) -- (44.4036,74.8193);
  \path[shift={(0,782.35975)},draw=ccccccc,line join=miter,line cap=butt,line
    width=0.800pt] (55.3012,74.8193) -- (164.6024,74.8193);
  \path[shift={(0,782.35975)},draw=ccccccc,line join=miter,line cap=butt,line
    width=0.800pt] (175.5000,74.8193) -- (256.9916,74.8193);
  \path[shift={(0,782.35975)},draw=ccccccc,line join=miter,line cap=butt,line
    width=0.800pt] (8.4737,78.0723) -- (60.3434,78.0723);
  \path[shift={(0,782.35975)},draw=ccccccc,line join=miter,line cap=butt,line
    width=0.800pt] (71.2410,78.0723) -- (180.5422,78.0723);
  \path[shift={(0,782.35975)},draw=ccccccc,line join=miter,line cap=butt,line
    width=0.800pt] (191.4398,78.0723) -- (256.9955,78.0723);
  \path[shift={(0,782.35975)},draw=ccccccc,line join=miter,line cap=butt,line
    width=0.800pt] (8.4698,81.0000) -- (74.9819,81.0000);
  \path[shift={(0,782.35975)},draw=ccccccc,line join=miter,line cap=butt,line
    width=0.800pt] (85.8795,81.0000) -- (195.1807,81.0000);
  \path[shift={(0,782.35975)},draw=ccccccc,line join=miter,line cap=butt,line
    width=0.800pt] (206.0783,81.0000) -- (256.9939,81.0000);
  \path[shift={(0,782.35975)},draw=ccccccc,line join=miter,line cap=butt,line
    width=0.800pt] (8.4721,84.2530) -- (95.1506,84.2530);
  \path[shift={(0,782.35975)},draw=ccccccc,line join=miter,line cap=butt,line
    width=0.800pt] (106.0482,84.2530) -- (215.3494,84.2530);
  \path[shift={(0,782.35975)},draw=ccccccc,line join=miter,line cap=butt,line
    width=0.800pt] (226.2470,84.2530) -- (256.9938,84.2530);
  \path[shift={(0,782.35975)},draw=ccccccc,line join=miter,line cap=butt,line
    width=0.800pt] (8.4654,89.7831) -- (70.4277,89.7831);
  \path[shift={(0,782.35975)},draw=ccccccc,line join=miter,line cap=butt,line
    width=0.800pt] (81.3253,89.7831) -- (190.6265,89.7831);
  \path[shift={(0,782.35975)},draw=ccccccc,line join=miter,line cap=butt,line
    width=0.800pt] (201.5241,89.7831) -- (257.0030,89.7831);
  \path[shift={(0,782.35975)},draw=ccccccc,line join=miter,line cap=butt,line
    width=0.800pt] (8.4682,97.5904) -- (97.4277,97.5904);
  \path[shift={(0,782.35975)},draw=ccccccc,line join=miter,line cap=butt,line
    width=0.800pt] (108.3253,97.5904) -- (217.6265,97.5904);
  \path[shift={(0,782.35975)},draw=ccccccc,line join=miter,line cap=butt,line
    width=0.800pt] (228.5241,97.5904) -- (256.9980,97.5904);
  \path[shift={(0,782.35975)},draw=ccccccc,line join=miter,line cap=butt,line
    width=0.800pt] (8.4707,106.6988) -- (67.8253,106.6988);
  \path[shift={(0,782.35975)},draw=ccccccc,line join=miter,line cap=butt,line
    width=0.800pt] (78.7229,106.6988) -- (188.0241,106.6988);
  \path[shift={(0,782.35975)},draw=ccccccc,line join=miter,line cap=butt,line
    width=0.800pt] (198.9217,106.6988) -- (257.0094,106.6988);
  \path[shift={(0,782.35975)},draw=ccccccc,line join=miter,line cap=butt,line
    width=0.800pt] (8.4687,118.0843) -- (110.1145,118.0843);
  \path[shift={(0,782.35975)},draw=ccccccc,line join=miter,line cap=butt,line
    width=0.800pt] (121.0121,118.0843) -- (230.3133,118.0843);
  \path[shift={(0,782.35975)},draw=ccccccc,line join=miter,line cap=butt,line
    width=0.800pt] (241.2108,118.0843) -- (256.9950,118.0843);
  \path[shift={(0,782.35975)},draw=ccccccc,line join=miter,line cap=butt,line
    width=0.800pt] (8.4637,138.9036) -- (68.8012,138.9036);
  \path[shift={(0,782.35975)},draw=ccccccc,line join=miter,line cap=butt,line
    width=0.800pt] (79.6988,138.9036) -- (189.0000,138.9036);
  \path[shift={(0,782.35975)},draw=ccccccc,line join=miter,line cap=butt,line
    width=0.800pt] (199.8976,138.9036) -- (257.0018,138.9036);
  \path[shift={(0,782.35975)},draw=ccccccc,line join=miter,line cap=butt,line
    width=0.800pt] (8.4639,161.6747) -- (109.1386,161.6747);
  \path[shift={(0,782.35975)},draw=ccccccc,line join=miter,line cap=butt,line
    width=0.800pt] (120.0361,161.6747) -- (229.3373,161.6747);
  \path[shift={(0,782.35975)},draw=ccccccc,line join=miter,line cap=butt,line
    width=0.800pt] (240.2349,161.6747) -- (256.9880,161.6747);
  \path[shift={(0,782.35975)},draw=ccccccc,line join=miter,line cap=butt,line
    width=0.800pt] (9.5964,182.1687) -- (118.8976,182.1687);
  \path[shift={(0,782.35975)},draw=ccccccc,line join=miter,line cap=butt,line
    width=0.800pt] (129.7952,182.1687) -- (239.0964,182.1687);
  \path[shift={(0,782.35975)},draw=ccccccc,line join=miter,line cap=butt,line
    width=0.800pt] (249.9940,182.1687) -- (256.9953,182.1687);
  \path[draw=black,dash pattern=on 19.20pt off 2.40pt,line join=miter,line
    cap=butt,miter limit=4.00,line width=2.400pt] (51.3976,820.8899) --
    (189.6506,820.8899);
  \path[fill=black] (78.397591,809.1308) node[above right] (text4468)
    {\Huge{$\vec{v}$}};
  \path[draw=black,fill=black,line join=miter,line cap=butt,line width=1.577pt]
    (171.8634,812.7573) -- (171.8634,830.3236) -- (194.9336,821.9981) -- cycle;
  \path[fill=black] (82.7111,844.0593) -- (86.2243,850.1443) -- (82.7781,856.1134)
    -- cycle;
  \path[fill=black] (12.165049,975.85052) node[above right] (text5217)
    {$\hat{x}$};
  \path[fill=black] (71.545792,865.97754) node[above right] (text5221)
    {$\hat{y}$};
  \path[shift={(0,782.35975)},draw=black,line join=miter,line cap=butt,miter
    limit=4.00,line width=1.600pt] (8.7831,198.7590) -- (8.7831,67.6627) --
    (85.2289,67.6627);
  \path[fill=black] (15.1355,980.6634) -- (9.0505,984.1766) -- (3.0814,980.7303)
    -- cycle;
  \path[fill=black] (42.168678,941.66095) node[above right] (text5766)
    {$\vec{v}=\vec{v}\left(t,y\right)=\begin{cases}0 & t<0\\v_{\infty} & x=0,\;
    t\geq0\\0 & x=\infty,\; t\geq0\end{cases}$};
\end{scope}

\end{tikzpicture}

\par\end{centering}

\protect\caption{Sudden Motion Problem Illustration}


\end{figure}



\subsection{Assumptions}
\begin{enumerate}
\item Adiabatic Isothermal (no Energy Equation)
\item Two Dimensional ($\frac{\partial}{\partial z}=0$)
\item Quasi Fully Developed ($\frac{\partial v_{y}}{\partial y}=0$)
\item Laminar ($\tau=-\mu\frac{\partial v_{y}}{\partial x}$)
\item Incompressible ($\frac{D\rho}{Dt}=0$)
\item Uniform Surface Pressure ($\left.p\right\rfloor _{y=0}=p_{\infty}$)
\end{enumerate}

\subsection{Continuity Equation}

\[
\frac{\partial\rho}{\partial t}+\nabla\cdot\left(\rho\vec{v}\right)=0\;\;\therefore\;\;\cancelto{0}{\frac{\partial\rho}{\partial t}+\vec{v}\cdot\nabla p}+\rho\nabla\cdot\vec{v}=0
\]


\[
\frac{\partial v_{x}}{\partial x}+\cancelto{0}{\frac{\partial v_{y}}{\partial y}}=0\;\;\therefore\;\;\frac{\partial v_{x}}{\partial x}=0
\]


\[
v_{x}=c_{1}\;\;\&\;\;\left.v_{x}\right\rfloor _{0}=0\;\;\therefore\;\;v_{x}=0
\]



\subsection{Momentum Equation}

\[
\frac{\partial\rho\vec{v}}{\partial t}+\nabla\cdot\left(\rho\vec{v}\vec{v}\right)=-\nabla p-\nabla\cdot\boldsymbol{\tau}+\rho\vec{g}
\]


and applying to the x direction

\begin{multline*}
\rho\left(\frac{\partial v_{x}}{\partial t}+\cancelto{0}{v_{x}}\frac{\partial v_{y}}{\partial x}+v_{y}\cancelto{0}{\frac{\partial v_{y}}{\partial x}}+v_{x}\cancelto{0}{\frac{\partial v_{y}}{\partial x}}\right)\\
=-\cancelto{0}{\frac{\partial p}{\partial y}}+\cancelto{0}{\rho g_{y}}+\mu\left(\frac{\partial^{2}v_{y}}{\partial x^{2}}+\cancelto{0}{\frac{\partial^{2}v_{y}}{\partial y^{2}}}+\cancelto{0}{\frac{\partial^{2}v_{y}}{\partial z^{2}}}\right)
\end{multline*}


so

\[
\frac{\partial v_{y}}{\partial t}=\frac{\mu}{\rho}\frac{\partial^{2}v_{y}}{\partial x^{2}}\;\;\&\;\;\nu\equiv\frac{\mu}{\rho}
\]


\[
\frac{\partial v_{y}}{\partial t}=\nu\frac{\partial^{2}v_{y}}{\partial x^{2}}
\]


Now use similarity solution to solve this instace of the heat equation
form.

\bibliographystyle{plain}
\bibliography{th}

\end{document}
