%% LyX 2.1.2 created this file.  For more info, see http://www.lyx.org/.
%% Do not edit unless you really know what you are doing.
\documentclass[english]{article}
\usepackage[T1]{fontenc}
\usepackage[latin9]{inputenc}
\usepackage[letterpaper]{geometry}
\geometry{verbose,tmargin=2cm,bmargin=2cm,lmargin=2cm,rmargin=2cm,headheight=1cm,headsep=1cm,footskip=1cm}
\usepackage{fancyhdr}
\pagestyle{fancy}
\setcounter{secnumdepth}{-1}
\usepackage{amsmath}

\makeatletter
%%%%%%%%%%%%%%%%%%%%%%%%%%%%%% User specified LaTeX commands.
\usepackage{pgf}
\usepackage{lastpage}

\let\oldmaketitle\maketitle

\renewcommand{\maketitle}{\oldmaketitle \thispagestyle{fancy}}

\makeatother

\usepackage{babel}
\begin{document}

\lhead{Alex Hagen}


\chead{QE Studying: Thermal Hydraulics Solutions}


\rhead{1/4/15}


\cfoot{\thepage\ of \pageref{LastPage}}


\title{Energy Equations \cite{Bird2007}}

\maketitle

\section{Energy Equation}

Applying the the GBE $\psi=\rho\left(u+\frac{\vec{v}^{2}}{2}\right)$,
$\boldsymbol{J}=\vec{q}+\boldsymbol{T}\vec{v}$, and $\dot{\psi_{g}}=\rho\vec{v}\cdot\vec{g}+\dot{q}$,
we get

\begin{multline*}
\underbrace{\frac{\partial\left(\rho\left(u+\frac{\vec{v}^{2}}{2}\right)\right)}{\partial t}}_{\substack{\text{rate of energy change}\\
\text{per unit volume}
}
}+\underbrace{\nabla\cdot\rho\vec{v}\left(u+\frac{\vec{v}^{2}}{2}\right)}_{\substack{\text{rate of energy change}\\
\text{by convection per unit}\\
\text{volume}
}
}\\
=-\underbrace{\nabla\cdot\vec{q}}_{\substack{\text{conduction}}
}-\underbrace{\nabla\cdot\left(p\vec{v}\right)}_{\substack{\text{work done by}\\
\text{pressure}
}
}-\underbrace{\nabla\cdot\left(\boldsymbol{\tau}\cdot\vec{v}\right)}_{\substack{\text{work done by}\\
\text{viscous force}
}
}+\underbrace{\rho\left(\vec{v}\cdot\vec{g}\right)}_{\substack{\text{work done}\\
\text{by gravity}
}
}+\underbrace{\dot{q}}_{\substack{\text{heat}\\
\text{generation}
}
}
\end{multline*}


to put this in a more usable form, we can do a couple steps to split
it into separate kinetic energy and internal energy forms. First,
we FOIL the flux terms:

\[
-\nabla\cdot\left(p\vec{v}\right)=-\underbrace{p\nabla\cdot\vec{v}}_{\substack{\text{reversible}\\
\text{\ensuremath{pdV} work}
}
}-\underbrace{\vec{v}\cdot\nabla p}_{\substack{\text{kinetic}\\
\text{energy}
}
}
\]


\[
-\nabla\cdot\left(\boldsymbol{\tau}\cdot\vec{v}\right)=-\underbrace{\boldsymbol{\tau}:\nabla\vec{v}}_{\substack{\text{irreversible}\\
\text{viscous work}
}
}-\underbrace{\vec{v}\cdot\nabla\cdot\boldsymbol{\tau}}_{\substack{\text{kinetic}\\
\text{energy}
}
}
\]


These create the two forms:
\begin{itemize}
\item Internal Energy:
\[
\frac{Du}{Dt}=-\nabla\cdot\vec{q}-p\nabla\cdot\vec{v}-\boldsymbol{\tau}:\nabla\vec{v}+\dot{q}
\]

\item Kinetic Energy:
\[
\frac{D\frac{\vec{v}^{2}}{2}}{Dt}=-\vec{v}\cdot\nabla p-\vec{v}\cdot\nabla\cdot\boldsymbol{\tau}+\rho\vec{v}\cdot\vec{g}
\]

\end{itemize}
These are not basic balance equations, though, because they don't
follow the form $\nabla\left(\,\;\;\;\right)$.

Other equations to remember:
\begin{itemize}
\item Poisson Heat Equation:
\[
\rho c_{p}\frac{\partial T}{\partial t}=\nabla\cdot\left(k\nabla T\right)+\dot{q}
\]

\item Temperature Equation:
\[
\rho c_{v}\frac{DT}{Dt}=\rho c_{v}\frac{\partial T}{\partial t}+\rho c_{v}\vec{v}\nabla\cdot T=k\nabla^{2}T-\boldsymbol{\tau}:\nabla\vec{v}+T\left.\frac{\partial p}{\partial T}\right)_{\rho}\nabla\cdot\vec{v}+\dot{q}
\]

\end{itemize}
\bibliographystyle{plain}
\bibliography{th}

\end{document}
