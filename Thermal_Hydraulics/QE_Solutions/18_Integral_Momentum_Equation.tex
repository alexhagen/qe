%% LyX 2.1.3dev created this file.  For more info, see http://www.lyx.org/.
%% Do not edit unless you really know what you are doing.
\documentclass[english]{article}
\usepackage[T1]{fontenc}
\usepackage[latin9]{inputenc}
\usepackage[letterpaper]{geometry}
\geometry{verbose,tmargin=2cm,bmargin=2cm,lmargin=2cm,rmargin=2cm,headheight=1cm,headsep=1cm,footskip=1cm}
\usepackage{fancyhdr}
\pagestyle{fancy}
\setcounter{secnumdepth}{-1}
\usepackage{float}
\usepackage{amsmath}
\usepackage{cancel}
\usepackage{esint}

\makeatletter
%%%%%%%%%%%%%%%%%%%%%%%%%%%%%% User specified LaTeX commands.
\usepackage{pgf}
\usepackage{tikz}
\usepackage{pstricks}
\usepackage{lastpage}

\makeatletter

\def\maketitle{%
\lhead{\@author}
\chead{\textsc{\@title}}
\rhead{\@date}
\rfoot{\thepage\ of \pageref{LastPage}}
\lfoot{QE Studying: Thermal Hydraulics Solutions}
\cfoot{}
\thispagestyle{fancy}
}

\makeatother

\newcommand\encircle[1]{%
  \tikz[baseline=(X.base)] 
    \node (X) [draw, shape=circle, inner sep=0] {\strut #1};}

\makeatother

\usepackage{babel}
\begin{document}

\title{Integral Momentum Equation}


\author{Alex Hagen}


\date{12/31/14}

\maketitle

\subsection{Setup}

There is differing area in the piping throughout the reactor system,
so 

\[
\rho_{i}v_{i}a_{i}=\rho_{i+1}v_{i+1}a_{i+1}=\rho_{r}v_{r}a_{r}
\]


and, assuming incompressible, but $p=p\left(t\right)$, we can write
this as

\[
v_{i}=\frac{a_{r}}{a_{i}}v_{r}
\]


Now we write the integral momentum equation

\[
\rho\frac{Dv}{Dt}=-\frac{\partial p}{\partial z}-\frac{f}{2D}\rho v\left|v\right|+\rho g_{z}-\rho\beta\Delta Tg_{z}
\]


And integrating over the entire loop

\[
\oint\left\{ \underbrace{\rho_{i}\frac{\partial v_{i}}{\partial t}}_{\encircle{1}}+\underbrace{\rho_{i}\frac{\partial v_{i}^{2}}{\partial z}}_{\encircle{2}}=-\underbrace{\frac{\partial\rho}{\partial z}}_{\encircle{3}}-\underbrace{\frac{f}{2D}\rho_{i}v_{i}^{2}}_{\encircle{4}}+\underbrace{\rho_{i}g_{zi}-\rho_{i}\beta\Delta T_{i}g_{zi}}_{\encircle{5}}\right\} dz
\]


Now, to find each term

\encircle{1}: THis term is not dependent on $z$ and so by integrating
and expanding the velocity, it becomes, simply:

\[
\oint\frac{\partial\rho_{i}v_{i}}{\partial t}dz=\rho_{r}\sum\left\{ \left(\frac{a_{r}}{a_{i}}\right)l_{i}\right\} \frac{\partial v_{r}}{\partial t}
\]


\encircle{2}: This term is cancelled out because any mass flow rate
change must be cancelled throughout system (or the area at the beginning
and end of a loop integral must be the same):

\[
\oint\frac{\partial\rho_{i}v_{i}^{2}}{\partial z}dz=\rho_{r}v_{r}^{2}a_{r}^{2}\left[\frac{1}{a_{i}}\right]_{\text{loop}}=0
\]


\encircle{3}: The pressure in the system will also be cancelled out,
except at a discontinuity, so this term goes to just the pumping pressure

\[
\oint\frac{\partial p}{\partial z}dz=0
\]


\[
\int\frac{df\left(x\right)}{dx}dx\ne f\left(x\right)\;\text{at a discontinuity}
\]


\[
\oint\frac{\partial p}{\partial z}dz=\Delta p_{pump}
\]


\encircle{4}: This term involves invoking the major and minor loss
form (as in the bernoulli Equation):

\[
\oint\frac{f_{i}\rho_{i}v_{i}\left|v_{i}\right|}{2D_{i}}dz=\sum_{i}\left\{ \left(\frac{fl}{D}+k\right)_{i}\left(\frac{a_{r}}{a_{i}}\right)^{2}\right\} \frac{\rho_{r}v_{r}^{2}}{2}
\]


\encircle{5}: By invoking the boussinesq assumption, and knowing
that the gravity will equal out through the loop, we can find only
the temperature difference term to stay in the equation:

\[
\oint\rho_{i}g_{zi}+\rho_{i}\beta\Delta T_{i}g_{zi}dz=\oint\cancelto{0}{\rho_{i}g_{zi}}+\rho_{i}\beta\cancelto{\Delta T_{h}}{\Delta T_{i}}g_{zi}dz=\rho_{r}\beta g\underbrace{\Delta T_{h}l_{h}}_{\substack{\text{heated}\\
\text{length and}\\
\text{temperature}\\
\text{change}
}
}
\]


so, putting this all together, we get

\[
\rho_{r}\frac{\partial v_{r}}{\partial t}\sum\left(\frac{a_{r}}{a_{i}}l_{i}\right)=-\Delta p_{pump}-\frac{\rho_{r}v_{r}^{2}}{2}\sum\left\{ \left(\frac{fl}{D}+k\right)_{i}\left(\frac{a_{r}}{a_{i}}\right)^{2}\right\} +\rho_{r}\beta g\Delta T_{h}l_{h}
\]


And we can split this into two different cases of use:


\subsection{Forced Convection}

For steady state, this is simple, as

\[
\Delta p_{pump}=-\frac{\rho_{r}v_{r}^{2}}{2}\sum\left\{ \left(\frac{fl}{D}+k\right)_{i}\left(\frac{a_{r}}{a_{i}}\right)^{2}\right\} 
\]


But if the transient happens, then

\[
\Delta p_{pump}=\Delta p_{pump}\left(t\right)
\]


and should follow the charts:

\begin{figure}[H]
\centering{}
\definecolor{ccccccc}{RGB}{204,204,204}


\begin{tikzpicture}[y=0.80pt, x=0.8pt,yscale=-1, inner sep=0pt, outer sep=0pt]
\begin{scope}[shift={(0,-782.35975)}]
  \path[color=black,fill=ccccccc,line width=1.600pt] (155.0313,811.9063) --
    (157.0313,811.9063) -- (157.0313,795.0938) -- (155.0313,795.0938) --
    cycle(155.0313,859.9063) -- (157.0313,859.9063) -- (157.0313,835.9063) --
    (155.0313,835.9063) -- cycle(155.0313,907.9063) -- (157.0313,907.9063) --
    (157.0313,883.9063) -- (155.0313,883.9063) -- cycle(155.0313,955.9063) --
    (157.0313,955.9063) -- (157.0313,931.9063) -- (155.0313,931.9063) --
    cycle(155.0313,1003.9062) -- (157.0313,1003.9062) -- (157.0313,979.9063) --
    (155.0313,979.9063) -- cycle;
  \path[fill=ccccccc] (199.85852,1044.8153) node[above right] (text3932) {Time
    ($t$) [s]};
  \path[cm={{0.0,-1.0,1.0,0.0,(0.0,0.0)}},fill=ccccccc] (-977.02356,19.215298)
    node[above right] (text3932-1) {\rotatebox{90}{Pressure Difference ($\Delta
    p$) [Pa]}};
  \path[fill=ccccccc] (36.135246,903.2829) node[above right] (text3847-2-5-2-1)
    {$0$};
  \path[draw=ccccccc,line join=miter,line cap=butt,miter limit=4.00,line
    width=1.600pt] (46.5621,899.3597) -- (354.3550,899.3597)(46.4594,794.8260) --
    (354.5406,794.8260) -- (354.5406,1003.8935) -- (46.4594,1003.8935) -- cycle;
  \path[shift={(0,782.35975)},draw=black,line join=miter,line cap=butt,line
    width=0.800pt] (46.2959,35.6805) -- (156.3550,35.6805) .. controls
    (156.3550,107.2331) and (199.5087,116.8935) .. (271.7101,116.8935) --
    (271.7101,168.3728) -- (353.8107,169.2604);
  \path[fill=ccccccc] (149.74971,1020.1519) node[above right] (text3847-2-5-2-1-4)
    {$t_{\text{trip}}$};
  \begin{scope}[fill=ccccccc]
    \path[shift={(0,782.35975)},color=black,fill=ccccccc,line width=1.200pt]
      (300.0000,115.3438) -- (296.3750,125.1875) .. controls (297.2458,124.5625) and
      (298.2388,124.1885) .. (299.2500,124.0625) -- (299.2500,133.8125) --
      (300.7500,133.8125) -- (300.7500,124.0625) .. controls (301.7579,124.1869) and
      (302.7394,124.5589) .. (303.5938,125.1875) -- (300.0000,115.3438) --
      cycle(299.2500,152.7813) -- (299.2500,161.1875) .. controls
      (298.2428,161.0629) and (297.2600,160.7220) .. (296.4063,160.0938) --
      (300.0000,169.9063) -- (303.6250,160.0938) .. controls (302.7548,160.7183) and
      (301.7605,161.0613) .. (300.7500,161.1875) -- (300.7500,152.7813) --
      (299.2500,152.7813) -- cycle;
  \end{scope}
  \path[fill=ccccccc] (279.53284,929.38678) node[above right]
    (text3847-2-5-2-1-4-8) {$\Delta p_{\text{flywheel}}$};
\end{scope}

\end{tikzpicture}
\protect\caption{Pump Transient Pressure Behavior}
\end{figure}


\begin{figure}[H]
\begin{centering}

\definecolor{ccccccc}{RGB}{204,204,204}


\begin{tikzpicture}[y=0.80pt, x=0.8pt,yscale=-1, inner sep=0pt, outer sep=0pt]
\begin{scope}[shift={(0,-782.35975)}]
  \path[color=black,fill=ccccccc,line width=1.600pt] (155.0313,811.9063) --
    (157.0313,811.9063) -- (157.0313,795.0938) -- (155.0313,795.0938) --
    cycle(155.0313,859.9063) -- (157.0313,859.9063) -- (157.0313,835.9063) --
    (155.0313,835.9063) -- cycle(155.0313,907.9063) -- (157.0313,907.9063) --
    (157.0313,883.9063) -- (155.0313,883.9063) -- cycle(155.0313,955.9063) --
    (157.0313,955.9063) -- (157.0313,931.9063) -- (155.0313,931.9063) --
    cycle(155.0313,1003.9062) -- (157.0313,1003.9062) -- (157.0313,979.9063) --
    (155.0313,979.9063) -- cycle;
  \path[fill=ccccccc] (199.85852,1044.8153) node[above right] (text3932) {Time
    ($t$) [s]};
  \path[cm={{0.0,-1.0,1.0,0.0,(0.0,0.0)}},fill=ccccccc] (-977.02356,19.215298)
    node[above right] (text3932-1) {\rotatebox{90}{Coolant Velocity ($v$)
    [$\mathrm{m}{s}$]}};
  \path[fill=ccccccc] (36.135246,903.2829) node[above right] (text3847-2-5-2-1)
    {$0$};
  \path[draw=ccccccc,line join=miter,line cap=butt,miter limit=4.00,line
    width=1.600pt] (46.5621,899.3597) -- (354.3550,899.3597)(46.4594,794.8260) --
    (354.5406,794.8260) -- (354.5406,1003.8935) -- (46.4594,1003.8935) -- cycle;
  \path[shift={(0,782.35975)},draw=black,line join=miter,line cap=butt,line
    width=0.800pt] (46.2959,35.6805) -- (156.3550,35.6805) .. controls
    (156.3550,51.2813) and (199.5087,58.8935) .. (271.7101,58.8935) --
    (287.8916,58.8935);
  \path[fill=ccccccc] (149.74971,1020.1519) node[above right] (text3847-2-5-2-1-4)
    {$t_{\text{trip}}$};
  \path[fill=ccccccc] (291.37622,878.32654) node[above right]
    (text3847-2-5-2-1-4-8) {\parbox[b]{0.65in}{if flywheel has inertia}};
  \path[color=black,fill=black,line width=0.800pt] (155.3125,818.1722) --
    (155.3750,819.9222) -- (155.3750,819.9534) -- (155.5938,821.6722) --
    (155.5938,821.7034) -- (155.9063,823.3596) -- (155.9063,823.3909) --
    (156.3750,825.0159) -- (156.3750,825.0471) -- (156.9375,826.6096) --
    (156.9375,826.6408) -- (157.6250,828.1720) -- (157.6250,828.2032) --
    (158.4063,829.6720) -- (158.4375,829.7032) -- (158.5312,829.8594) --
    (159.3749,829.3282) -- (159.3124,829.2032) -- (159.2812,829.1720) --
    (158.5312,827.7657) -- (158.5312,827.7345) -- (157.8750,826.2970) --
    (157.8438,826.2345) -- (157.3125,824.7033) -- (156.8750,823.1720) --
    (156.9062,823.1720) -- (156.5625,821.5470) -- (156.5625,821.5158) --
    (156.3750,819.8908) -- (156.3750,819.8596) -- (156.3125,818.1096) --
    cycle(166.6250,838.8285) -- (166.7500,838.9222) -- (168.2813,840.0784) --
    (169.8750,841.1722) -- (169.9062,841.1722) -- (171.5625,842.2347) --
    (173.2812,843.2659) -- (173.3124,843.2659) -- (175.0937,844.2347) --
    (176.9062,845.1722) -- (177.3749,844.2659) -- (175.5624,843.3597) --
    (173.8124,842.3909) -- (173.7812,842.3909) -- (172.0937,841.3909) --
    (172.0625,841.3909) -- (170.4375,840.3284) -- (170.4375,840.3596) --
    (168.8750,839.2659) -- (167.3750,838.1409) -- (167.2500,838.0472) --
    cycle(188.0625,849.7660) -- (189.0625,850.1097) -- (189.0937,850.1097) --
    (193.4062,851.4847) -- (197.8125,852.7347) -- (199.5937,853.2035) --
    (199.8437,852.2347) -- (198.0625,851.7660) -- (198.0625,851.7972) --
    (193.6875,850.5472) -- (189.4062,849.1722) -- (188.4062,848.7972) --
    cycle(211.3438,855.8285) -- (215.6875,856.6410) -- (215.7187,856.6410) --
    (220.0625,857.3597) -- (220.0937,857.3597) -- (223.1874,857.8285) --
    (223.3437,856.8285) -- (220.2499,856.3597) -- (220.2187,856.3597) --
    (215.8750,855.6410) -- (211.5312,854.8597) -- cycle(235.1250,859.2660) --
    (236.2500,859.3910) -- (243.0938,859.9847) -- (243.1250,859.9847) --
    (247.0937,860.2660) -- (247.1562,859.2660) -- (243.1875,858.9847) --
    (236.3750,858.3910) -- (236.3438,858.3910) -- (235.2188,858.2973) --
    cycle(259.1250,860.6097) -- (271.1250,860.6097) -- (271.1250,859.6097) --
    (259.1250,859.6097) -- cycle(283.1250,860.6097) -- (288.2937,860.6097) --
    (288.2937,859.6097) -- (283.1250,859.6097) -- cycle;
  \path[fill=ccccccc] (291.79199,843.84497) node[above right]
    (text3847-2-5-2-1-4-8-2) {ideal};
\end{scope}

\end{tikzpicture}

\par\end{centering}

\protect\caption{Pump Transient Velocity Behavior}
\end{figure}



\subsection{Natural Convection}

We must first determine the temperature change created, so

\[
\rho c_{p}\left\{ \cancelto{0}{\frac{\partial T}{\partial t}}+v_{r}\frac{\partial T}{\partial z}\right\} =\frac{\xi_{h}q_{h}^{''}}{a_{r}}
\]


\[
\frac{\partial T}{\partial z}=\frac{\xi_{h}q_{h}^{''}}{\rho c_{p}a_{r}v_{r}}
\]


\[
\Delta T_{h}=\frac{\xi_{h}q_{0}^{''}l_{core}}{\rho c_{p}a_{r}v_{r}}
\]


And plugging this in, we get

\[
\frac{\rho_{r}v_{r}^{2}}{2}\sum\left\{ \left(\frac{fl}{D}+k\right)_{i}\left(\frac{a_{r}}{a_{i}}\right)^{2}\right\} =\rho_{r}\beta g\left(\frac{\xi_{h}q_{0}^{''}l_{core}}{\rho c_{p}a_{r}v_{r}}\right)l_{h}
\]


\[
v_{r}=\sqrt[3]{\frac{\xi_{h}q_{0}^{''}l_{core}l_{h}\beta g}{\frac{\rho c_{p}a_{r}}{2}\sum\left\{ \left(\frac{fl}{D}+k\right)_{i}\left(\frac{a_{r}}{a_{i}}\right)^{2}\right\} }}
\]


\bibliographystyle{plain}
\bibliography{th}

\end{document}
