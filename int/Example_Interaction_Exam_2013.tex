%% LyX 2.1.3dev created this file.  For more info, see http://www.lyx.org/.
%% Do not edit unless you really know what you are doing.
\documentclass[english]{article}
\usepackage[T1]{fontenc}
\usepackage[latin9]{inputenc}
\usepackage{geometry}
\geometry{verbose,tmargin=2cm,bmargin=2cm,lmargin=2cm,rmargin=2cm,headheight=1cm,headsep=1cm,footskip=1cm}
\usepackage{float}
\usepackage{amssymb}
\usepackage[authoryear]{natbib}
\PassOptionsToPackage{normalem}{ulem}
\usepackage{ulem}

\makeatletter
%%%%%%%%%%%%%%%%%%%%%%%%%%%%%% User specified LaTeX commands.
\usepackage[noanswer]{exercise}
\usepackage{mhchem}
\usepackage{graphicx}
\usepackage{color}
\usepackage{transparent}
\renewcommand{\ExerciseListHeader}{\textbf{\ExerciseHeaderNB) \ExerciseHeaderTitle \ExerciseHeaderOrigin}}
\renewcommand{\AnswerListHeader}{\textbf{Answer of \ExerciseName \ExerciseHeaderNB\  -- }}
\renewcommand{\ExePartListHeader}{\ExePartHeaderNB)}
\@ifpackagewith{exercise}{noanswer}{%
  \setlength{\Exesep}{4 cm}}{%
  \setlength{\Exetopsep}{1 em}}
\setlength{\Exepartopsep}{1 em}

\renewenvironment{Exercise}{\begin{minipage}{\pagewidth}}{\end{minipage}}
\renewcommand{\theExePart}{\alph{ExePart}}
\renewcommand{\ExePartListHeader}{\emph{\ExePartHeaderDifficulty\ExePartHeaderNB)}}

\makeatother

\usepackage{babel}
\begin{document}

\title{Example Interactions Exam 2013}


\author{Alex Hagen}

\maketitle
\begin{ExerciseList}

\interlinepenalty=10000

\Exercise[%
  year={2013},%
  exam={QE-I},%
  label={wavelength-energy-conversion},%
  title={Conversion from Wavelength to Energy},%
  difficulty={4},%
  origin={\cite[p. 56, prob. 2.27]{Cember2012a}}]

\ExeText What is the wavelength of 

\ExePart \quad{}an electron whose kinetic energy is \sout{$1000\,eV$}?
($1\,MeV$ - changed by A.Hagen, January 2014)

\ExePart \quad{}\sout{a $10^{-8}\,kg$ oil droplet falling at
the rate of $0.01\,\frac{m}{s}$?}

\ExePart \quad{}a $1\,MeV$ neutron?

\Answer[%
  ref={wavelength-energy-conversion}]

We must convert first the energy to (relativistic) momentum, which
for mass bearing particles is related to $\lambda$ by $\lambda=\frac{h}{p}$
and for non-mass bearing particles $\lambda=h\nu$ \citet[p. 14]{Lamarsh2001}.

\ExePart \quad{}The electron in this case is relativistic. The rest
mass of a neutron is $0.511\,MeV$. We use the equation \citet[p. 15]{Lamarsh2001}
\[
\lambda=\frac{hc}{\sqrt{E_{total}^{2}-E_{rest}^{2}}}=\frac{1240\,eV\cdot nm}{\sqrt{\left(1\,MeV+0.511\,MeV\right)-\left(0.511\,MeV\right)^{2}}}=8.72\times10^{-13}\,nm
\]


\ExePart \quad{}We do not solve this part since it won't be asked
on the exam.

\ExePart \quad{}The neutron in this case is not relativistic ($1\,MeV<0.2\cdot m_{rest}=20\,MeV$),
nevertheless we use the same equation

\[
\lambda=\frac{hc}{\sqrt{E_{total}^{2}-E_{rest}^{2}}}=\frac{1240\,eV\cdot nm}{\sqrt{\left(1\,MeV+938\,MeV\right)-\left(938\,MeV\right)^{2}}}=2.86\times10^{-14}\,nm
\]


\Exercise[%
  year={2013},%
  exam={QE-I},%
  label={positron-electron-pair},%
  title={Annihilation of Positron-Electron Pair},%
  difficulty={4},%
  origin={unfound}]

A $5\,MeV$ electron interacts with a positron at rest. What interaction
will occur? What will the resulting types and energies of the two
resulting particles be?

\Answer[%
  ref={positron-electron-pair}]

First we must determine what kind of interaction will happen, which
in this case is \textbf{annihilation}. This will interact as the diagram
given below:

\begin{figure}[H]
\centering \def\svgwidth{10 cm} \input{annihilation_diagram.pdf_tex}

\protect\caption{Annihilation of Energetic Electron onto Resting Positron}


\end{figure}


First, we can start an energy balance.

\[
E_{e^{-}}+E_{e^{+}}=E_{\gamma1}+E_{\gamma2}
\]


\[
\left(5\,MeV+0.511\,MeV\right)+0.511\,MeV=E_{\gamma1}+E_{\gamma2}\;\therefore\;E_{\gamma1}+E_{\gamma2}=6.022\,MeV
\]


We know that the two $\gamma$'s will head off in opposite directions,
so we can do a momentum balance.

\[
p_{e^{-}}+p_{e^{+}}=p_{\gamma1}-p_{\gamma2}
\]


Using the definition for momentum of $\gamma$'s:

\[
p_{e^{-}}+p_{e^{+}}=\frac{E_{\gamma1}}{c}-\frac{E_{\gamma2}}{c}
\]


and using the definition for relativistic particle momentum, and that
the stationary positron has no momentum

\[
p_{e^{-}}\cdot c=E_{\gamma1}-E_{\gamma2}
\]


\[
\left(\frac{1}{c}\sqrt{E_{total}^{2}-E_{rest}^{2}}\right)\cdot c=E_{\gamma1}-E_{\gamma2}
\]


\[
\sqrt{\left(5\,MeV\right)^{2}-\left(0.511\,MeV\right)^{2}}=E_{\gamma1}-E_{\gamma2}\;\therefore\;E_{\gamma1}-E_{\gamma2}=5.49\,MeV
\]


with two above definitions

\[
E_{\gamma1}=
\]


\[
E_{\gamma2}=
\]


\Exercise[%
  year={2013},%
  exam={QE-I},%
  label={radioactive-decay-99mo-99tc},%
  title={Radioactive Decay and Production},%
  difficulty={4},%
  origin={\cite[p. 137]{Cember2012a}}]

$37\,MBq$ ($1\,mCi$) \ce{^{99m}Tc} are ``milked'' from a \ce{^{99}Mo}
``cow.'' What will be the activity of the \ce{^{99m}Tc} daughter,
\ce{^{99}Tc}, $1\,yr$ after the milking \citep{Cember2012a}?

Half Lives from \citep{KoreaAtomicEnergyResearchInstitute2000}

\[
T_{\frac{1}{2}^{99m}Tc}=6.01\,hr,\;T_{\frac{1}{2},^{99}Mo}=65.94\,hr,\;T_{\frac{1}{2}^{99}Tc}=2.11\times10^{5}\,yr
\]


\Answer[%
  ref={radioactive-decay-99mo-99tc}]

First we must assume that the \ce{^{99m}Tc} is removed from the \ce{^{99}Mo}
at $t=0$, so all we have is \ce{^{99m}Tc} decaying to \ce{^{99}Tc}.
In this case, we can write two differential equations. The first has
to do with the amount of \ce{^{99m}Tc} atoms over time

\[
\frac{dN_{1}}{dt}=-\lambda_{1}N_{1}
\]


which can easily be solved to show

\[
N_{1}(t)=N_{1}(0)\cdot\exp\left(-\lambda_{1}t\right)
\]


Now we have a second differential equation for the amount of \ce{^{99}Tc}
atoms over time

\[
\frac{dN_{2}}{dt}=-\lambda_{2}N_{2}+\lambda_{1}N_{1}
\]


by solving this and subsituting in the solution to the first differential
equation, we get \{lamarsh 26\}

\[
N_{2}=N_{2}(0)\exp\left(-\lambda_{2}t\right)+\frac{N_{1}(0)\lambda_{2}}{\lambda_{2}-\lambda_{1}}\left(\exp\left(-\lambda_{1}t\right)-\exp\left(-\lambda_{2}t\right)\right)
\]


\Exercise[%
  year={2013},%
  exam={QE-I},%
  label={alpha-particles-on-detector-geometry},%
  title={Alpha Particle Beam onto Specified Detector Geometry},%
  difficulty={7},%
  origin={unfound}]

We cannot find an appropriate exercise similar to this problem.

\Exercise[%
  year={2013},%
  exam={QE-I},%
  label={stopping-power-deuteron-proton-alpha},%
  title={Calculation of the Stopping Power of a Particle},%
  difficulty={4},%
  origin={\cite[p. 64, prob 2.12]{Knoll2000}}]

A beam of $1\,MeV$ electrons strikes a thick target. For a beam current
of $100\,\mu A$, find the power dissipated in the target \citep{Knoll2000}.

\Answer[%
  ref={stopping-power-deuteron-proton-alpha}]

The intensity can be found by

\[
I=100\times10^{-6}\,\frac{C}{s}\cdot\frac{1\,e^{-}}{1.6\times10^{-19}\,C}=6.25\times10^{14}\,\frac{e^{-}}{s}
\]


and, assuming that the entirity of the $1\,MeV$ is dissipated in
the target on each collision, the power dissipated is

\[
P=\frac{1\,MeV}{e^{-}}\cdot\frac{6.25\times10^{14}\,e^{-}}{s}=6.25\times10^{14}\,\frac{MeV}{s}=100.1\,W
\]


\end{ExerciseList}

\vfill{}


\bibliographystyle{plainnat}
\bibliography{/home/ahagen/bibs/QE}

\end{document}
